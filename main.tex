\documentclass[aspectratio=169, usepdftitle=false, xcolor={dvipsnames}]{beamer}

\makeatletter
\appto\input@path{{pkgs/awesome-beamer}, {pkgs/smile}}
\makeatother

\definecolor{forest}{HTML}{004225}

\usetheme[english, coloraccent=forest, notoc]{awesome}

\usetikzlibrary{tikzmark}
\makeatletter
\tikzset{
	contoured/.style={preaction={draw=white,line width=2.5\smile@linewidth}},
	gn/.style={roundednode, fill=lightgray, draw=none, inner sep=1.2mm, minimum width=2em},
	dot/.style={circle,fill,inner sep=1pt},
	outline/.style={roundednode, draw=accent, anchor=north east, outer sep=1ex, inner sep=1ex},
	read/.style={arrow,draw=red!40!gray},
	def/.style={arrow,draw=blue!40!gray},
	occ/.style={arrow,draw=green!40!gray},
	con/.style={arrow,dashed,draw=gray}
}

\newtcolorbox{dottedborder}{enhanced,frame empty,interior empty,boxsep=\z@,borderline={\smile@linewidth}{\z@}{gray,dashed}}
\makeatother

\setminted{
	numbers=none,
	frame=none
}
\usemintedstyle{stata}

\title[\insertsubtitle]{Graph it like it's hot}
\author{Lukas Pietzschmann}
\subtitle{Building a data flow graph for java}
\email{lukas.pietzschmann@uni-ulm.de}
\institute{Institute of Software Engineering and Programming Languages}
\uni{University of Ulm}
\location{Ulm}
\date{10/29/2023}
% \background{background.png}

\begin{document}

\maketitle

\section{Code Example}
\begin{frame}[fragile]
	\frametitle\secname
	\newsavebox\steps\makeatletter
	\savebox\steps{\begin{tikzpicture}[node distance=1ex]
		\node[inline] (On)   [opacity=0.3] {\mintinline{java}{dependsOn(...)}};
		\node[inline] (From) [below=of On.south east,anchor=north east] {\mintinline{java}{flowsFrom(...)}};
		\node[inline] (To)   [below=of From.south east,anchor=north east,opacity=0.3] {\mintinline{java}{flowsTo(...)}};

		\node[gn,xshift=2mm] (L1) [right=of To] {};
		\node[gn] (L2) [right=of L1, xshift=1ex] {};
		\coordinate (C) at ($(L1)!0.5!(L2)$);
		\node[gn] (R) at (C |- On) {};
		\foreach \i in {L1,L2} \draw[arrow, line cap=round, shorten >=1mm,shorten <= 1mm] (R) to (\i);

		\node[gn,xshift=2mm] (L21) [right=of L2] {};
		\node[gn] (L22) [right=of L21, xshift=1ex] {};
		\coordinate (C2) at ($(L21)!0.5!(L22)$);
		\node[gn] (R2) at (C2 |- R) {};

		\node[dot] (RW) at (R2.west) {};
		\node[dot] (RE) at (R2.east) {};
		\node[dot] (RSE) at ([xshift=-3mm]R2.south east) {};
		\node[dot] (RSW) at ([xshift=3mm]R2.south west) {};
		\node[dot,xshift=-1pt,yshift=-1pt] (L1E) at (L21.north east) {};
		\node[dot,xshift=1pt,yshift=-1pt] (L1W) at (L21.north west) {};
		\node[dot,xshift=-1pt,yshift=-1pt] (L2E) at (L22.north east) {};
		\node[dot,xshift=1pt,yshift=-1pt] (L2W) at (L22.north west) {};

		\draw[arrow,dashed] (RW) to[out=180,in=90] node[roundnode,inner sep=0.3pt,solid]{\tiny1} (L1W);
		\draw[arrow,dashed] (L1E) to node[roundnode,inner sep=0.3pt,solid]{\tiny2} (RSW);
		\draw[arrow,dashed] (RSE) to node[roundnode,inner sep=0.3pt,solid]{\tiny3}(L2W);
		\draw[arrow,dashed] (L2E) to[out=90,in=0] node[roundnode,inner sep=0.3pt,solid]{\tiny4} (RE);

		\coordinate (C3) at ($(To.east)!0.5!(L1.west)$);
		\coordinate (C4) at ($(L2.east)!0.5!(L21.west)$);
		\foreach \i in {C3,C4} \draw[gray,line width=\smile@linewidth,line cap=round] (\i |- On.north) to (\i |- To.south);
	\end{tikzpicture}}
	\begin{columns}[c]
		\begin{column}{0.55\textwidth}
			\begin{dottedborder}
			\begin{minted}[escapeinside=||, fontsize=\scriptsize]{java}
				class AlDeSCo {
					public int magic(int |\tikzmark{n}|n, boolean b) {
						int x = 42;
						if(b)
							x = addFive(n);
						else
							x = n;
						return x;
					}
					private int addFive(int m) {
						return m + 5;
					}
				}
			\end{minted}
			\end{dottedborder}
		\end{column}
		\begin{column}{0.35\textwidth}
			\block How does \tikzmark{t}\mintinline{java}{n} propagate through the program?\endblock\bigskip
			\centerline{\clap{\scalebox{.85}{\smile@linewidth0.7px\usebox\steps}}}
		\end{column}
	\end{columns}
	\tikz[overlay,remember picture]\draw[textarrow, contoured] ([xshift=0.5ex,yshift=1ex]pic cs:t) to[in=-90, out=90] ([xshift=0.5ex,yshift=-0.1ex]pic cs:n);
\end{frame}

\section{Select the Operation}
\begin{frame}
	\frametitle\secname
	\newsavebox\stepi
	\savebox\stepi{\begin{tikzpicture}[node distance=1ex]
		\node[inline] (On)   [opacity=0.3] {\mintinline{java}{dependsOn(...)}};
		\node[inline] (From) [below=of On.south east,anchor=north east] {\mintinline{java}{flowsFrom(...)}};
		\node[inline] (To)   [below=of From.south east,anchor=north east,opacity=0.3] {\mintinline{java}{flowsTo(...)}};
	\end{tikzpicture}}
	\tikz[overlay,remember picture]\node[outline] at (current page.north east) {\usebox\stepi};
	\begin{columns}[c]
		\begin{column}{0.6\textwidth}
			We can ask three different questions:
			\begin{itemize}
				\item Does \texttt{x} \textbf{depend on} \texttt{y}?
				\item What \textbf{flows from} \texttt{x}?
				\item What \textbf{flows to} \texttt{x}?
			\end{itemize}
		\end{column}
		\begin{column}{0.3\textwidth}
			\setlength\fboxsep{0px}\setlength\fboxrule{1px}%
			\textcolor{accent}{\fbox{\href{https://imgflip.com/memegenerator/184932428/Three-Take-it-or-leave-it}{\includegraphics[width=\textwidth]{imgs/patrick.png}}}}
		\end{column}
	\end{columns}
\end{frame}

\section{Build the dataflow graph}
\begin{frame}
	\frametitle\secname
	\newsavebox\stepii
	\savebox\stepii{\begin{tikzpicture}[node distance=1ex]
		\node[gn,xshift=2mm] (L1) {};
		\node[gn] (L2) [right=of L1, xshift=1ex] {};
		\node[gn,yshift=1cm] (R) at ($(L1)!0.5!(L2)$) {};
		\foreach \i in {L1,L2} \draw[arrow, line cap=round, shorten >=1mm,shorten <= 1mm] (R) to (\i);
	\end{tikzpicture}}
	\tikz[overlay,remember picture]\node[outline] at (current page.north east) {\usebox\stepii};
	\begin{tikzpicture}[node distance=4ex]
		\node[gn] (narg)                     {n\strut};
		\node[gn,xshift=4cm] (n) [right=of narg] {n\strut};
		\coordinate (C) at ($(narg)!0.5!(n)$);
		\coordinate[above=of n] (T);
		\node[gn] (intn) at (C |- T)         {int n\strut};
		\node[gn] (addFive) [below=of narg]  {addFive(n)\strut};
		\node[gn] (xthen) [below=of addFive] {x = addFive(n)\strut};
		\node[gn] (xelse) at (n |- xthen)    {x = n\strut};
		\coordinate[below=of xelse] (B);
		\node[gn] (returnx) at (C |- B)      {return x\strut};
		\node[gn] (intx) at (C |- xelse)     {int x = 42\strut};
		\node[gn] (b) at (C |- addFive)      {b\strut};
		\node[gn] (boolb) at (C |- n)        {boolean b\strut};

		\foreach \i in {xthen,xelse} {
			\draw[read] (returnx) to (\i);
			\draw[occ] (\i) to (intx);
			\draw[con] (\i) to (b);
		}
		\foreach \i in {n,narg} \draw[read] (\i) to (intn);
		\draw[read] (addFive) to (narg);
		\draw[def] (xthen) to (addFive);
		\draw[def] (xelse) to (n);
		\draw[read] (b) to (boolb);
	\end{tikzpicture}
	\begin{tikzpicture}[node distance=4ex]
		\node[gn] (intm) {int m\strut};
		\node[gn] (m) [below=of intm] {m\strut};
		\node[gn] (e) [below=of m] {m + 5\strut};

		\draw[def] (e) to (m);
		\draw[read] (m) to (intm);
	\end{tikzpicture}
	\begin{tikzpicture}[node distance=4ex]
		\node[gn] (add) {addFive\strut};
		\node[gn] (magic) [below=of add] {magic\strut};

		\draw[arrow] (magic) to (add);
	\end{tikzpicture}
\end{frame}

\section{Traverse the dataflow graph}
\begin{frame}
	\frametitle\secname
	\newsavebox\stepiii
	\savebox\stepiii{\begin{tikzpicture}[node distance=1ex]
		\node[gn,xshift=2mm] (L21) {};
		\node[gn] (L22) [right=of L21, xshift=1ex] {};
		\node[gn,yshift=1cm] (R2) at ($(L21)!0.5!(L22)$) {};

		\node[dot] (RW) at (R2.west) {};
		\node[dot] (RE) at (R2.east) {};
		\node[dot] (RSE) at ([xshift=-3mm]R2.south east) {};
		\node[dot] (RSW) at ([xshift=3mm]R2.south west) {};
		\node[dot,xshift=-1pt,yshift=-1pt] (L1E) at (L21.north east) {};
		\node[dot,xshift=1pt,yshift=-1pt] (L1W) at (L21.north west) {};
		\node[dot,xshift=-1pt,yshift=-1pt] (L2E) at (L22.north east) {};
		\node[dot,xshift=1pt,yshift=-1pt] (L2W) at (L22.north west) {};

		\draw[arrow,dashed] (RW) to[out=180,in=90] node[roundnode,inner sep=0.3pt,solid]{\tiny1} (L1W);
		\draw[arrow,dashed] (L1E) to node[roundnode,inner sep=0.3pt,solid]{\tiny2} (RSW);
		\draw[arrow,dashed] (RSE) to node[roundnode,inner sep=0.3pt,solid]{\tiny3}(L2W);
		\draw[arrow,dashed] (L2E) to[out=90,in=0] node[roundnode,inner sep=0.3pt,solid]{\tiny4} (RE);
	\end{tikzpicture}}
	\tikz[overlay,remember picture]\node[outline] at (current page.north east) {\usebox\stepiii};
\end{frame}

\end{document}