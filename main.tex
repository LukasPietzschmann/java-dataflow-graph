\documentclass[aspectratio=169, usepdftitle=false, xcolor={dvipsnames}]{beamer}

\makeatletter
\appto\input@path{{pkgs/awesome-beamer}, {pkgs/smile}}
\makeatother

\definecolor{forest}{HTML}{004225}

\usetheme[english, coloraccent=forest, notoc]{awesome}

\usetikzlibrary{tikzmark}
\makeatletter
\tikzset{
	vf/.style={alt={<#1->{}{invisible}}},
	short/.style n args={1}{shorten <=#1,shorten >=#1},
	contoured/.style={preaction={draw=white,line width=2.5\smile@linewidth}},
	gn/.style={roundednode, fill=lightgray, draw=none, inner sep=1.2mm, minimum width=2em},
	dot/.style={circle,fill,inner sep=1pt},
	outline/.style={roundednode, draw=accent, anchor=north east, outer sep=1ex, inner sep=1ex,line width=\smile@linewidth},
	doutline/.style={roundednode, draw=gray, dashed, node on layer=background, outer sep=1ex, inner sep=1ex,line width=\smile@linewidth},
	cr/.style={line cap=round},
	read/.style={cr,short=1mm,arrow,draw=red!40!gray},
	def/.style={cr,short=1mm,arrow,draw=blue!40!gray},
	occ/.style={cr,short=1mm,arrow,draw=green!40!gray},
	con/.style={cr,short=1mm,arrow,dashed,draw=gray}
}

\newtcolorbox{dottedborder}{enhanced,frame empty,interior empty,boxsep=\z@,borderline={\smile@linewidth}{\z@}{gray,dashed}}
\makeatother

\setminted{
	numbers=none,
	frame=none
}
\usemintedstyle{algol_nu}

\colorlet{green}{accent}

\title[\insertsubtitle]{Graph it like it's hot}
\author{Lukas Pietzschmann}
\subtitle{Building a data flow graph for java}
\email{lukas.pietzschmann@uni-ulm.de}
\institute{Institute of Software Engineering and Programming Languages}
\uni{University of Ulm}
\location{Ulm}
\date{10/29/2023}
% \background{background.png}

\begin{document}

\maketitle

\section{Code Example}
\begin{frame}[fragile]
	\frametitle\secname
	\newsavebox\steps\makeatletter
	\savebox\steps{\begin{tikzpicture}[node distance=1ex]
		\node[inline] (On)   [opacity=0.3] {\mintinline{java}{dependsOn(...)}};
		\node[inline] (From) [below=of On.south east,anchor=north east] {\mintinline{java}{flowsFrom(...)}};
		\node[inline] (To)   [below=of From.south east,anchor=north east,opacity=0.3] {\mintinline{java}{flowsTo(...)}};

		\node[gn,xshift=2mm] (L1) [right=of To] {};
		\node[gn] (L2) [right=of L1, xshift=1ex] {};
		\coordinate (C) at ($(L1)!0.5!(L2)$);
		\node[gn] (R) at (C |- On) {};
		\foreach \i in {L1,L2} \draw[arrow, cr, short=1mm] (R) to (\i);

		\node[gn,xshift=2mm] (L21) [right=of L2] {};
		\node[gn] (L22) [right=of L21, xshift=1ex] {};
		\coordinate (C2) at ($(L21)!0.5!(L22)$);
		\node[gn] (R2) at (C2 |- R) {};

		\node[dot] (RW) at (R2.west) {};
		\node[dot] (RE) at (R2.east) {};
		\node[dot] (RSE) at ([xshift=-3mm]R2.south east) {};
		\node[dot] (RSW) at ([xshift=3mm]R2.south west) {};
		\node[dot,xshift=-1pt,yshift=-1pt] (L1E) at (L21.north east) {};
		\node[dot,xshift=1pt,yshift=-1pt] (L1W) at (L21.north west) {};
		\node[dot,xshift=-1pt,yshift=-1pt] (L2E) at (L22.north east) {};
		\node[dot,xshift=1pt,yshift=-1pt] (L2W) at (L22.north west) {};

		\draw[arrow,dashed] (RW) to[out=180,in=90] node[roundnode,inner sep=0.3pt,solid]{\tiny1} (L1W);
		\draw[arrow,dashed] (L1E) to node[roundnode,inner sep=0.3pt,solid]{\tiny2} (RSW);
		\draw[arrow,dashed] (RSE) to node[roundnode,inner sep=0.3pt,solid]{\tiny3}(L2W);
		\draw[arrow,dashed] (L2E) to[out=90,in=0] node[roundnode,inner sep=0.3pt,solid]{\tiny4} (RE);

		\coordinate (C3) at ($(To.east)!0.5!(L1.west)$);
		\coordinate (C4) at ($(L2.east)!0.5!(L21.west)$);
		\foreach \i in {C3,C4} \draw[gray,line width=\smile@linewidth,cr] (\i |- On.north) to (\i |- To.south);
	\end{tikzpicture}}
	\begin{columns}[c]
		\begin{column}{0.55\textwidth}
			\begin{dottedborder}
			\begin{minted}[escapeinside=||, fontsize=\scriptsize]{java}
				class AlDeSCo {
					public int magic(int |\tikzmark{n}|n, boolean b) {
						int x = 42;
						if(b)
							x = addFive(n);
						else
							x = n;
						return x;
					}
					private int addFive(int m) {
						return m + 5;
					}
				}
			\end{minted}
			\end{dottedborder}
		\end{column}
		\begin{column}{0.35\textwidth}
			\block How does \tikzmark{t}\mintinline{java}{n} propagate through the program?\endblock\bigskip
			\centerline{\clap{\scalebox{.85}{\smile@linewidth0.7px\usebox\steps}}}
		\end{column}
	\end{columns}
	\tikz[overlay,remember picture]\draw[textarrow, contoured] ([xshift=0.5ex,yshift=1ex]pic cs:t) to[in=-90, out=90] ([xshift=0.5ex,yshift=-0.1ex]pic cs:n);
\end{frame}

\section{Select the Operation}
\begin{frame}
	\frametitle\secname
	\newsavebox\stepi
	\savebox\stepi{\begin{tikzpicture}[node distance=1ex]
		\node[inline] (On)   [opacity=0.3] {\mintinline{java}{dependsOn(...)}};
		\node[inline] (From) [below=of On.south east,anchor=north east] {\mintinline{java}{flowsFrom(...)}};
		\node[inline] (To)   [below=of From.south east,anchor=north east,opacity=0.3] {\mintinline{java}{flowsTo(...)}};
	\end{tikzpicture}}
	\tikz[overlay,remember picture]\node[outline] at (current page.north east) {\usebox\stepi};
	\begin{columns}[c]
		\begin{column}{0.6\textwidth}
			We can ask three different questions:
			\begin{itemize}
				\item Does \texttt{x} \textbf{depend on} \texttt{y}?
				\item What \textbf{flows from} \texttt{x}?
				\item What \textbf{flows to} \texttt{x}?
			\end{itemize}
		\end{column}
		\begin{column}{0.3\textwidth}
			\setlength\fboxsep{0px}\setlength\fboxrule{1px}%
			\textcolor{accent}{\fbox{\href{https://imgflip.com/memegenerator/184932428/Three-Take-it-or-leave-it}{\includegraphics[width=\textwidth]{imgs/patrick.png}}}}
		\end{column}
	\end{columns}
\end{frame}

\section{Build the dataflow graph}
\newsavebox\stepii\newsavebox\legend
\savebox\stepii{\begin{tikzpicture}[node distance=1ex]
	\node[gn,xshift=2mm] (L1) {};
	\node[gn] (L2) [right=of L1, xshift=1ex] {};
	\node[gn,yshift=1cm] (R) at ($(L1)!0.5!(L2)$) {};
	\foreach \i in {L1,L2} \draw[arrow, cr, short=1mm] (R) to (\i);
\end{tikzpicture}}
\savebox\legend{\begin{tikzpicture}[node distance=1ex]
	\node[gn] (A1) {\rlap{a}\phantom{b}};
	\node[gn, right, right=of A1,xshift=1cm] (B1) {b};
	\node[gn, below, below=of A1,yshift=-5mm] (A2) {\rlap{a}\phantom{b}};
	\node[gn, right, right=of A2,xshift=1cm] (B2) {b};
	\node[gn, below, right=of B1] (A3) {\rlap{a}\phantom{b}};
	\node[gn, right, right=of A3,xshift=1cm] (B3) {b};
	\node[gn, below, right=of B2] (A4) {\rlap{a}\phantom{b}};
	\node[gn, right, right=of A4,xshift=1cm] (B4) {b};
	\draw[read] (A1) to node[yshift=-5mm,red!30!black]{read from} (B1);
	\draw[def] (A2) to node[yshift=-5mm,blue!30!black]{def'ed by} (B2);
	\draw[occ] (A3) to node[yshift=-5mm,green!30!black]{occurrence} (B3);
	\draw[con] (A4) to node[yshift=-5mm,gray!30!black]{controlled by} (B4);
\end{tikzpicture}}
\begin{frame}[fragile]
	\frametitle\secname
	\begin{tikzpicture}[overlay, remember picture,node distance=-1ex]
		\node[outline] at (current page.north east) (S) {\usebox\stepii};
		\node[outline, left=of S.north west,anchor=north east] {\scalebox{.528}{\usebox\legend}};
	\end{tikzpicture}
	\begin{columns}[c]
		\begin{column}{0.45\textwidth}
			\tikzset{every node/.append style={font=\scriptsize,execute at begin node=\llap{\phantom b}},node distance=3ex}
			\begin{tikzpicture}
				\node[vf=11,gn] (narg)                          {n};
				\node[vf=20,gn,xshift=4cm] (n) [right=of narg]  {n};
				\coordinate (C) at ($(narg)!0.5!(n)$);
				\coordinate[above=of n] (T);
				\node[vf=4,gn,anchor=south] (intn) at (C |- T) {int n};
				\node[vf=12,gn] (addFive) [below=of narg]       {addFive(n)};
				\node[vf=18,gn] (xthen) [below=of addFive]      {x = addFive(n)};
				\node[vf=21,gn] (xelse) at (n |- xthen)         {x = n};
				\coordinate[below=of xelse] (B);
				\node[vf=23,gn, anchor=north](returnx)at(C |- B){return x};
				\node[vf=7,gn] (intx) at (C |- xelse)          {int x = 42};
				\node[gn] (b) at (C |- addFive)           {b};
				\node[vf=5,gn] (boolb) at (C |- n)             {boolean b};

				\foreach \i in {xthen,xelse} {
					\draw[read] (returnx) to (\i);
					\draw[occ] (\i) to (intx);
					\draw[con] (\i) to (b);
				}
				\foreach \i in {n,narg} \draw[read] (\i) to (intn);
				\draw[read] (addFive) to (narg);
				\draw[def] (xthen) to (addFive);
				\draw[def] (xelse) to (n);
				\draw[read] (b) to (boolb);

				\node[doutline,fit=(intn)(returnx)(xthen)(xelse)] {};
			\end{tikzpicture}
			% \begin{tikzpicture}
			% 	\node[gn] (intm) {int m};
			% 	\node[gn] (m) [below=of intm] {m};
			% 	\node[gn] (e) [below=of m] {m + 5};
			% 	\draw[def] (e) to (m);
			% 	\draw[read] (m) to (intm);
			% 	\node[doutline,fit=(intm)(m)(e)] {};
			% \end{tikzpicture}
			% \begin{tikzpicture}
			% 	\node[gn] (add) {addFive};
			% 	\node[gn] (magic) [below=of add] {magic};
			% 	\draw[arrow] (magic) to (add);
			% 	\node[doutline,fit=(add)(magic)] {};
			% \end{tikzpicture}
		\end{column}
		\begin{column}{0.45\textwidth}\def\m#1{\tikzmark{#1}}
			\begin{dottedborder}
			\begin{minted}[escapeinside=||, fontsize=\tiny]{java}
				|\m{cstart}|class AlDeSCo {
					|\m{mstart}|public int magic(|\m{ns}|int n|\m{ne}|, |\m{bs}|boolean b|\m{be}|) {
						|\m{intxs}|int x|\m{intxe}| = 42|\m{intxee}|;
						|\m{ifs}|if(b)
							|\m{thens}|x|\m{xte}| = |\m{adds}|addFive|\m{addse}|(|\m{pns}|n|\m{pne}|)|\m{adde}|;|\m{thene}|
						else
						|\m{ife}|	|\m{elses}|x|\m{xee}| = |\m{nes}|n|\m{nee}|;|\m{elsee}|
						|\m{rs}|return |\m{rxs}|x|\m{rxe}|;|\m{re}|
					|\m{mend}|}
					|\m{m2s}|private int addFive(|\m{ms}|int m|\m{me}|) {
						|\m{r2s}|return |\m{res}|m|\m{rme}| + 5|\m{ree}|;|\m{r2e}|
					|\m{m2e}|}
				|\m{cend}|}
			\end{minted}
			\end{dottedborder}
			\newcommand<>\ul[4][0pt]{\uncover#5{\tikz[remember picture, overlay]\fill[#2,opacity=0.3,rounded corners=.7px,visible on=#5] ([yshift=#1-0.8px]pic cs:#3) rectangle ([yshift=#1+1.2px]pic cs:#4);}}
			\newcommand<>\uls[4][0pt]{\uncover#5{\tikz[remember picture, overlay]\fill[#2,opacity=0.3,rounded corners=.7px,visible on=#5] ([xshift=#1-1px,yshift=0.5ex]pic cs:#3) rectangle ([xshift=#1+1px]pic cs:#4);}} \uls<2-24>{red}{cstart}{cend}%   class
			\uls<3-23>{blue}{mstart}{mend}%       magic
			\ul<4>{blue}{ns}{ne}%                 paramN
			\ul<5>{blue}{bs}{be}%                 paramB
			\ul<6-7>{blue}{intxs}{intxee}%        int x = 42;
			\ul<7>[4px]{blue}{intxs}{intxe}%      int x
			\uls<8-21>{blue}{ifs}{ife}%           if-else
			\ul<9-18>{blue}{thens}{thene}%        then
			\ul<10-17>[4px]{blue}{adds}{adde}%    addFive(n)
			\ul<11>[6.5px]{blue}{pns}{pne}%       n
			\ul<12-17>[6.5px]{blue}{adds}{addse}% addFive
			\ul<18>[4px]{blue}{thens}{xte}%       x
			\ul<19-21>{blue}{elses}{elsee}%       else
			\ul<20>[4px]{blue}{nes}{nee}%         n
			\ul<21>[4px]{blue}{elses}{xee}%       x
			\ul<22-23>{blue}{rs}{re}%             return x;
			\ul<23>[4px]{blue}{rxs}{rxe}%         x

			\uls<13-17>{green}{m2s}{m2e}%         addFive
			\ul<14>{green}{ms}{me}%               paramM
			\ul<15-17>{green}{r2s}{r2e}%          return m + 5;
			\ul<16-17>[4px]{green}{res}{ree}%     m + 5
			\ul<17>[6.5px]{green}{res}{rme}%      m
		\end{column}
	\end{columns}
\end{frame}

\section{Traverse the dataflow graph}
\begin{frame}
	\frametitle\secname
	\newsavebox\stepiii
	\savebox\stepiii{\begin{tikzpicture}[node distance=1ex]
		\node[gn,xshift=2mm] (L21) {};
		\node[gn] (L22) [right=of L21, xshift=1ex] {};
		\node[gn,yshift=1cm] (R2) at ($(L21)!0.5!(L22)$) {};

		\node[dot] (RW) at (R2.west) {};
		\node[dot] (RE) at (R2.east) {};
		\node[dot] (RSE) at ([xshift=-3mm]R2.south east) {};
		\node[dot] (RSW) at ([xshift=3mm]R2.south west) {};
		\node[dot,xshift=-1pt,yshift=-1pt] (L1E) at (L21.north east) {};
		\node[dot,xshift=1pt,yshift=-1pt] (L1W) at (L21.north west) {};
		\node[dot,xshift=-1pt,yshift=-1pt] (L2E) at (L22.north east) {};
		\node[dot,xshift=1pt,yshift=-1pt] (L2W) at (L22.north west) {};

		\draw[arrow,dashed] (RW) to[out=180,in=90] node[roundnode,inner sep=0.3pt,solid]{\tiny1} (L1W);
		\draw[arrow,dashed] (L1E) to node[roundnode,inner sep=0.3pt,solid]{\tiny2} (RSW);
		\draw[arrow,dashed] (RSE) to node[roundnode,inner sep=0.3pt,solid]{\tiny3}(L2W);
		\draw[arrow,dashed] (L2E) to[out=90,in=0] node[roundnode,inner sep=0.3pt,solid]{\tiny4} (RE);
	\end{tikzpicture}}
	\tikz[overlay,remember picture]\node[outline] at (current page.north east) {\usebox\stepiii};
\end{frame}

\end{document}